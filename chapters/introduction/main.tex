\chapter{Introduction}

Read the thesis guidelines used at LAB University of Applied Sciences. Regarding
writing the thesis report, read especially chapter 6, Thesis Report’s language,
style, illustrations, structure and layout, and chapter 7, Data collection and
referencing mechanics. Follow the guidelines carefully.

This is the template used for writing thesis reports at LAB. Good skills in
using a word processor help in writing a thesis report and using the thesis
template. Therefore, before you start the writing process, it is advisable to
ensure that you have sufficient basic skills in editing long documents with a
word processor. These skills include the use of the Styles tool and an
understanding of document sections and automatic references. Using these tools
when writing a thesis report will probably save more hours than studying them
takes.

To understand how to use this thesis template, you need to have internalised the
following basics regarding MS Word:

\begin{itemize}
  \item No text is positioned, nor any layout done, by adding multiple
  consecutive spaces or line or paragraph breaks. If you need to press Enter or
  Space bar more than once in succession, you are probably doing something
  wrong. When starting a new paragraph, press Enter once at the end of the line,
  and the space between the paragraphs is added using styles.
  \item It not advisable to produce any numbered data (paragraph numbers, page
  numbering, numbering of images/tables/figures/appendices) by manually entering
  the numbers. For all of these, Word provides powerful automated tools that
  keep numbering organised, even if you edit, add or delete data.
  \item Never add hyphens to words at the end of a line by manually typing them.
  In the thesis template, Word’s automatic hyphenation is enabled. If you need
  to add hyphens yourself, you can use the optional hyphen tool in Word.
\end{itemize}

If you notice any errors in the thesis template, feel free to fix them.
